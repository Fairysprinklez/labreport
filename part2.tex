\section{General LaTeX tips}\label{sec:latex_tips}

\subsection{Matrix Equations}
Here is a matrix equation you can use as a template:
\begin{equation}
\begin{bmatrix}
 1 &  0 &  0 & 0 & -b &  0 &  0 &  0 \\
-a &  1 &  0 & 0 &  0 & -b &  0 &  0 \\
 0 & -a &  1 & 0 &  0 &  0 & -b &  0 \\
 0 &  0 & -a & 1 &  0 &  0 &  0 & -b                                
\end{bmatrix}
\begin{bmatrix} x_1 \\ x_2 \\ x_3 \\ x_4 \\ u_0 \\ u_1 \\ u_2 \\ u_3 \end{bmatrix}
=
\begin{bmatrix}
ax_0 \\ 0 \\ 0 \\ 0      
\end{bmatrix}
\end{equation}

\subsection{Tables}
If you want, you can use the source for Table~\ref{tab:parameters} to see how a (floating) table is made. 

Variables and symbols are always in italics, while units are not.

\begin{table}[tbp]
	\centering
	\caption{Parameters and values.}
	\begin{tabular}{llll}
		\hline
		Symbol & Parameter & Value & Unit \\
		\hline
		$l_a$ & Distance from elevation axis to helicopter body & $0.63$ & \meter \\
		$l_h$ & Distance from pitch axis to motor & $0.18$ & \meter \\
		$K_f$ & Force constant motor & $0.25$ & \newton\per\volt \\
		$J_e$ & Moment of inertia for elevation & $0.83$ & \kilogram\usk\meter\squared \\
		$J_t$ & Moment of inertia for travel & $0.83$ & \kilogram\usk\meter\squared \\
		$J_p$ & Moment of inertia for pitch & $0.034$ & \kilogram\usk\meter\squared \\
		$m_h$ & Mass of helicopter & $1.05$ & \kilogram \\
		$m_w$ & Balance weight & $1.87$ & \kilogram \\
		$m_g$ & Effective mass of the helicopter & $0.05$ & \kilogram \\
		$K_p$ & Force to lift the helicopter from the ground & $0.49$ & \newton \\
		\hline
	\end{tabular}
	\label{tab:parameters}
\end{table}


\subsection{The \texttt{\textbackslash{input}\{\}} command}
By using \texttt{\textbackslash{input}\{whatever\}} in your main tex file (\texttt{main.tex} in this case), the content of \texttt{whatever.tex} will be included in your pdf. This way you can split the contents into different files, e.g.~one for each problem of the assignment. This makes it easier to restructure the document, and arguably improves the readability of the tex files. For instance; maybe you want each problem to start on a new page? Simply add \textbackslash{newpage} before each \texttt{\textbackslash{input}} command.

\subsection{Citations}
In academic writing, it is important to credit your sources. In \LaTeX{} this is done by the \texttt{\textbackslash{cite}} command. For instance \texttt{\textbackslash{cite}\{Chen2014\}} will produce~\cite{Chen2014}, and make the full details of that source available in the references. This requires that you make a bibliography file (\texttt{bibliography.bib} in this case), containing something like
\lstinputlisting[language=Tex, firstline=1, lastline=7]{bibliography.bib}

There are many different citation styles, and a lot of customization that is possible, so please check out e.g.~\cite{WikibookLatex}\footnote{Even though this cites a web page, scientific writing tries to keep the citation of web pages to a minimum.}.

