% vim: ft=tex
\section{Useful packages}
This section lists some packages that are useful for reports. See also the comments in the preamble of this document.

\begin{description}
  \item[amsmath] provides some useful mathematical definitions (like \texttt{\textbackslash{sin}} for e.g. $\sin(\omega t)$ and \texttt{\textbackslash{lim}} for e.g. $\lim_{t\to\infty}$. Also some definitions to make it more convenient to write the math.
  \item[amssymb] provides an extended symbol collection.
  \item[hyperref] makes it possible to create ``clickable'' links, like \url{itk.ntnu.no/fag/TTK4115}, but also \cref{tab:extab}, \cref{eq:mmot} and \cref{eq:mmot2}.
  \item[graphicx] provides some useful definitions for dealing with graphics, like \textbackslash{textwidth}: \textbackslash{includegraphics}[width=\textbackslash{textwidth}]\{picture\} will have the same width as the text column.
  \item[float] lets you force placement of floats with the [H] option. You should normally not need this.
\end{description}
\begin{table}[H]
  \caption{Example table}
  \centering
  \begin{tabular}{ll}
    \hline
    \textbf{X} & \textbf{Y}\\
    \hline
    5 & 8\\
    6 & 9\\
    7 & 10 \\
    \hline
  \end{tabular}
  \label{tab:extab}
\end{table}

\begin{subequations}
  \begin{align}
    J_p\ddot{p} &= L_{1}V_{d} \label{eq:mmot1}\\
    J_e\ddot{e} &= L_{2} \cos(e) + L_3 V_s \cos(p) \label{eq:mmot2}\\
    J_\lambda \ddot{\lambda} &= L_4 V_s \cos(e) \sin(p) \label{eq:mmot3}
  \end{align}
  \label{eq:mmot}
\end{subequations}

More information on these packages can be found in the documentation for each package. The following subsections include some other packages that require a more in-depth explaination.

\subsection{listings}
The \texttt{listings} package makes it easy to include code in the report. For example \cref{lst:label} includes code that is written in the tex file. However \cref{lst:listing2} simply takes the code directly from the source file. You can also specify what the code listings should look like: color, line numbers, frames\ldots

\begin{lstlisting}[caption={Some Matlab code, with the source in the tex file},label={lst:label},language=Matlab, float]
  degree = 6;
  out = ones(size(X1(:,1)));
  for i = 1:degree
  for j = 0:i
  out(:, end+1) = (X1.^(i-j)).*(X2.^j);
  end
  end
\end{lstlisting}
\lstinputlisting[language=Matlab, firstline=4, lastline=10, caption={Another piece of code, directly from the source file}, label={lst:listing2},float]{some_function.m}
This is great! However, try to keep the amount of code in the report to a reasonable level, and remember; code in itself is not an explanation.

See \url{http://ctan.mirrors.hoobly.com/macros/latex/contrib/listings/listings.pdf} for more information.

\subsection{todonotes}
The \texttt{todonotes} package is great for work in progress. Few things are more embarrassing than forgetting to remove ``LALALALAL FIXME!!!!!!'' from the middle of your report. Instead, use \texttt{\textbackslash{todo}\{LALALA FIXME!!!\}} \todo{LALALA FIXME!!!}. This will show up like a red box in the margin. Some prefer \texttt{\textbackslash{todo}{[inline]}\{FIXME2!!!\}} which produces \todo[inline]{FIXME2!!!} To avoid typing [inline] all the time, you can define
\begin{lstlisting}[language=TeX, numbers=none]
  \newcommand{\TODO}[1]{\todo[inline]{#1}}
\end{lstlisting}
in the preamble of the document, and then use \textbackslash{TODO}.

And the best part: When you are finished with your report (or have run out of time) you can simply change \textbackslash{usepackage}\{todonotes\} to \textbackslash{usepackage}[disable]\{todonotes\} and they will all magically disappear!

You can also use \textbackslash{listoftodos} to get a list of all the todos in your document, and \textbackslash{missingfigure} will create a dummy figure, like \cref{fig:my_awesome_fig}, that you can replace once you have made a proper figure. This way you can start referencing figures/plots before you make them, and still be reminded that you need to make them.

See \url{http://ctan.math.utah.edu/ctan/tex-archive/macros/latex/contrib/todonotes/todonotes.pdf} for more information.
\begin{figure}[h]
  \centering
  \missingfigure{Drawing of such and such}
  \caption{Sweet figure, bro!}
  \label{fig:my_awesome_fig}
\end{figure}

\subsection{cleveref}
Speaking of referencing: The observant reader might have noticed the use of \texttt{\textbackslash{cref}} in referencing tables, figures etc. This is a bit more clever than the normal \texttt{\textbackslash{ref}} because it detects what you are referencing based on the prefix of the label. Then it prints the appropriate ``prefix''. So \texttt{\textbackslash{cref}\{fig:my\_awesome\_fig\}} will produce \cref{fig:my_awesome_fig}, whereas \texttt{\textbackslash{cref}\{tab:extab\}} will produce \cref{tab:extab}. Notice how the labels of the table and the figureare prefixed with \texttt{tab:} and \texttt{fig:} respectively. If you want it to say e.g. ``figure'' instead of ``fig.'', this is completely customizable. There is also \texttt{\textbackslash{Cref}} for a capitalized version.

See \url{http://mirrors.ibiblio.org/CTAN/macros/latex/contrib/cleveref/cleveref.pdf} for more information.
